\documentclass{article}
\usepackage[utf8]{inputenc}

\title{Tabla}
\author{aleexxpf }
\date{October 2021}

\begin{document}

\begin{table}[]
\begin{tabular}{lll}
\hline
\multicolumn{1}{|l|}{Algoritmo}  & \multicolumn{1}{l|}{Peor Caso}                                                                                                                                            & \multicolumn{1}{l|}{\begin{tabular}[c]{@{}l@{}}Orden de Complejidad \\ (tiempo, en el peor caso)\end{tabular}} \\ \hline
\multicolumn{1}{|l|}{Inserción}  & \multicolumn{1}{l|}{\begin{tabular}[c]{@{}l@{}}Cuando la lista está  ordenada \\ de manera descendente.\end{tabular}}                                                     & \multicolumn{1}{l|}{$$O(n^{2})$$}                                                           \\ \hline
\multicolumn{1}{|l|}{Selección}  & \multicolumn{1}{l|}{\begin{tabular}[c]{@{}l@{}}No hay peor ni mejor caso,\\ siempre va a realizar las \\ mismas operaciones\end{tabular}}                                 & \multicolumn{1}{l|}{$$O(n^{2})$$}                                                           \\ \hline
\multicolumn{1}{|l|}{Burbuja}    & \multicolumn{1}{l|}{\begin{tabular}[c]{@{}l@{}}Cuando la lista está  ordenada \\ de manera descendente.\end{tabular}}                                                     & \multicolumn{1}{l|}{$$O(n^{2})$$}                                                           \\ \hline
\multicolumn{1}{|l|}{Mezcla}     & \multicolumn{1}{l|}{\begin{tabular}[c]{@{}l@{}}No hay pero ni mejor caso, \\ siempre hace lo mismo, su \\ mayor desventaja es sobre la\\ memoria\end{tabular}}            & \multicolumn{1}{l|}{$$n \space log_2 n$$}                                                      \\ \hline
\multicolumn{1}{|l|}{Rápido}     & \multicolumn{1}{l|}{\begin{tabular}[c]{@{}l@{}}Cuando tomamos  como \\ pivote al primer o al \\ último elemento.\end{tabular}}                                            & \multicolumn{1}{l|}{$$O(n^{2})$$}                                                           \\ \hline
\multicolumn{1}{|l|}{Bae(Radix)} & \multicolumn{1}{l|}{\begin{tabular}[c]{@{}l@{}}No hay mejor ni peor caso\\ ya que siempre realizará \\ las mismas aginaciones\end{tabular}}                               & \multicolumn{1}{l|}{$$O(n)$$}                                                                                  \\ \hline
\multicolumn{1}{|l|}{Cubeta}    & \multicolumn{1}{|l|}{\begin{tabular}[c]{@{}l@{}}Cuando hay elementos que \\ son muy aproximados entre \\ si, además cuando también \\ hay muchos elementos en una \\ sola cubeta.\end{tabular}} & \multicolumn{1}{l|}{$$O(n^{2})$$}      
  \\ \hline

\end{tabular}
\end{table}

\end{document}

\documentclass{article}
\usepackage[utf8]{inputenc}

\title{tabla examen 2}
\author{aleexxpf }
\date{November 2021}

\begin{document}

\begin{table}[]
\begin{tabular}{|c|c|c|}
\hline
Tipo de árbol                                                       & Peor Caso                                                                                                 & \begin{tabular}[c]{@{}c@{}}Orden de complejidad\\ (Búsqueda)\end{tabular}                  \\ \hline
Árbol binario                                                       & \begin{tabular}[c]{@{}c@{}}Cuando generamos el\\ árbol degenerado\end{tabular}                            & \begin{tabular}[c]{@{}c@{}}(Depende de como \\ busquemos)\\ $O(n)$ (en general)\end{tabular} \\ \hline
\begin{tabular}[c]{@{}c@{}}Árbol binario\\ de búsqueda\end{tabular} & \begin{tabular}[c]{@{}c@{}}Cuando se forma el \\ árbol degenerado\end{tabular}                            & $O(n)$                                                                                       \\ \hline
\begin{tabular}[c]{@{}c@{}}Árbol binario \\ balanceado\end{tabular} & \begin{tabular}[c]{@{}c@{}}No hay peor caso, gracias \\ al FE todos los casos \\ son iguales\end{tabular} & $O(log_2 n)$                                                                                \\ \hline
\end{tabular}
\end{table}

\end{document}
